\documentclass{article}
\usepackage{graphicx} % Required for inserting images
\usepackage{amsfonts}
\usepackage{amsmath, amsthm}

\newcommand{\R}{\mathbb{R}}
\newtheorem{definition}{Definition}
\newtheorem{theorem}{Theorem}
\newtheorem{prop}[theorem]{Proposition}
\newtheorem{exercise}{Exercise}
\newtheorem{corollary}[theorem]{Corollary}
\newtheorem{lemma}[theorem]{Lemma}
\newtheorem{example}{Example}

\title{Victoir-Fritz}
\author{Alessandro Combi}
\date{June 2025}

\begin{document}

\maketitle

\section{Continuous Paths of bounded variation}

\subsection{Continuous paths on metric spaces}

A version of the Ascoli–Arzelà theorem is shown that characterizes relative compactness of families of continuous paths \( f:[0,T] \to E \), where \( E \) is a complete metric space in which bounded subsets have compact closure.

\begin{lemma}
Any continuous mapping from $f:[0,T] \rightarrow E$ is uniformly continuous if $E$ is a metric space.
\end{lemma}

\begin{lemma}
    $C([0,T],E)$ is metric under $|\cdot|_{\infty}$ and it is complete if $E$ is complete.
\end{lemma}

\begin{lemma}
    A subset of a complete metric space has compact closure if and only if it is totally bounded (that is for any $\varepsilon > 0$ it is covered by a finite number of $\varepsilon-$balls).
\end{lemma}

\begin{theorem}[Ascoli-Arzelà]
    Let $E$ be a complete metric space, where bounded sets have compact closure. Then a subset $H$ of $C([0,T],E)$ is bounded and equicontinuous if and only if it has compact closure.
\end{theorem}
\begin{proof}[Outline of the proof]
\begin{itemize}

\item Equicontinuity $\Rightarrow$ Compact closure.
The image through the whole $H$ of each element of a $\delta$-partition is in a finite union of $\varepsilon$-balls.
Hence I can divide $H$ based on mapping between elements of the partition and $\varepsilon$-balls.

\item Compact closure $\Rightarrow$ Equicontinuity.

Use total boundedness and pick the continuity module of the centers of $\varepsilon$-balls.
\end{itemize}
\end{proof}

\subsection{Continuous paths of bounded variation on
metric spaces}

\subsubsection{Bounded variation of paths and controls}

The notion of \textit{controls} is introduced, which is useful to bound the 1-variation of a path. Lower semi-continuity of 1-variation under pointwise convergence is shown.

\begin{definition}
    A map $\omega: \Delta_T = \{(s,t):\ 0 \leq s \leq t \leq T \} \rightarrow [0, \infty)$
    is said to be a control if it is subadditive, continuous and zero on the diagonal. 
\end{definition}

\begin{example}
    \begin{itemize}
        \item $(s,t) \mapsto |t - s|^{\theta},$ $\theta \geq 1$
        \item integral of a non-negative $L^1$ function over $[s,t].$
        \item Let $\phi \in C([0,\infty), [0,\infty))$ increasing and convex, with $\phi(0)=0.$ If $\omega$ is a control, then $\phi \omega$ is a control.
        \item $\omega \tilde{\omega}$ is a control,
        \item $max(\omega, \tilde{\omega})$ is not always a control,
        \item If $\alpha + \beta \geq 1,$ then $\omega^{\alpha} \tilde{\omega}^{\beta}$ is a control.
    \end{itemize}
\end{example}

\begin{prop}
    Let $x:[0,T] \rightarrow E$ and $\omega$ superadditive. If $d(x_s, x_t) \leq \omega(s, t)\ \forall s < t,$ then $|x|_{1-var; [s,t]} \leq \omega(s,t).$
\end{prop}

\begin{prop}
    Let $x \in C^{1-var}([0,T],E).$ Then $|x|_{1-var; [s,t]}$ defines a control and is additive.    
\end{prop}
\begin{proof}[Outline of the proof]
    Additivity is easy.
    To show continuity, one can define 1-variation "from inside" as $|x|_{1-var; [s^+,t^-]}$ and show it is also a control.
\end{proof}
\begin{lemma}
    Assume $(x^n)$ is a sequence of paths in $C^{1-var}([0,T],E)$. If $x^n \rightarrow x$ pointwise, then for all $s < t$ we have:
    \begin{equation}
        |x|_{1-var; [s,t]} \leq \liminf_{n \rightarrow \infty} |x^n|_{1-var; [s,t]}.
    \end{equation}
    The inequality above can be strict.
\end{lemma}
\begin{proof}[Outline of the proof]
    A counterexample is given by a sequence of functions oscillating $n$ times with oscillation size $1/n$.
\end{proof}
\subsubsection{Absolute continuity}

\begin{prop}[mistake in the book proof]
    Any absolutely continuous path is a continuous path of bounded variation.
\end{prop}

\begin{example}
    Cantor function: construct and show that it is a counterexample of the above (bounded variation, not absolutely continuous).
\end{example}

\subsection{Lipschitz and 1-Holder continuity}

1-Holder continuity is introduced, it is shown how finite 1-variation maps are reparametrizations of 1-Holder maps and the lower semicontinuity of 1-Holder norm under pointwise convergence is shown.

\begin{lemma}[not a lemma in the book]\label{lem:HolderControl}
    A path $x$ is 1-Holder if and only if it is controlled by $(s,t) \mapsto |t-s|.$
    Furthermore, 
    \begin{equation}
    |x|_{1-var,[s,t]} \leq |x|_{1-Hol,[s,t]}|s-t|.
    \end{equation}
\end{lemma}
\begin{proof}[Outline of proof]
Restrict $x$ to $[s,t]$ and control it with $(u,v) \mapsto |x|_{1-Hol,[s,t]}|u-v|.$
\end{proof}

\begin{prop}[Reparametrization]
    A path $x \in C([0,T], E)$ is of finite $1$-variation if and only if there exist a continuous non-decreasing function $\phi:[0,T] \rightarrow [0,1]$ and an $y \in C^{1-Hol}([0,1], E)$ such that $y\phi= x.$
\end{prop}
\begin{proof}[Outline of the proof]
    Choose $\phi(t) = \frac{|x|_{1-var,[0,t]}}{|x|_{1-var,[0,T]}}$.
    One can additionally prove that that $|x|_{1-var,[0,T]}=|y|_{1-Hol,[0,1]}$.
\end{proof}

\begin{lemma}
    Assume $(x^n)$ is a sequence of paths from $[0,T] \rightarrow E$ of finite $1$-variation. Assume $x^n \rightarrow x$ pointwise. Then for all $s < t$ we have:
    \begin{equation}
        |x|_{1-Hol, [s,t]} \leq \liminf_{n \rightarrow \infty} |x^n|_{1-Hol, [s,t]}
    \end{equation}
\end{lemma}
\begin{proof}[Outline of the proof]
    Follows from the lower semicontinuity of the 1-variation norm and from Lemma \ref{lem:HolderControl}.
\end{proof}

\subsection{Continuous paths of bounded variation in $\R^d$}

\subsubsection{Continuously differentiable paths}

\begin{prop}[1-variation of $C^1$ paths]
    Let $x \in C^1([0,T], \R^d).$ Then 
    \begin{equation}
        t \in [0,T] \rightarrow l(t) := |x|_{1-var; [0,t]}
    \end{equation}
    is continuously differentiable with $\dot{l} = |\dot{x}|.$
    In particular $|x|_{1-var, [s,t]} = \int_s^t |\dot{x}_u| du.$
\end{prop}

\subsubsection{Bounded variation}

\begin{theorem}
    $C^{1-var}([0,T], \R^d)$ is Banach with norm $|x(0)| + |x|_{1-var, [0,T]}.$ It is not separable.
\end{theorem}

\begin{example}
    Example of uncountable set of 1-variation functions in $C^{1-var}([0,T], \R)$ for which $|f_\alpha - f_{\alpha'}|_{1-var} \geq 1$ if $\alpha \neq \alpha'.$
    One can define a sinusoidal $\alpha_n$ oscillating $n$ times in $[1-\frac{1}{2^{n-1}}, 1 - \frac{1}{2^n}]$, divided by $2n.$ 
    Then the uncountable set is formed by the sums of $\alpha_n$ where $n$ is in a specific subset of $\mathbb{N}.$
\end{example}

\begin{prop}[mistake in the book]
    Let $x \in C^{1-var}([0,T], \R^d).$ Then for any dissection $D$ and any $s < t$ in the dissection $D:$
    \begin{equation}
        |x^D|_{1-var; [s,t]} \leq |x|_{1-var; [s,t]}.
    \end{equation}

    If $|D_n| \rightarrow 0$, $x^{D_n}$ converges uniformly to $x$.
\end{prop}

\begin{prop}
    The set of absolutely continuous functions from $[0,T] \rightarrow \R^d$ is closed in 1-variation and a Banach space under 1-variation norm.
\end{prop}

\begin{example}
    Denote by $D_n = \{ j 3^{-n}:\ j = 0, \ldots, 3^n \}.$ Then if $f$ is the Cantor function we have:
    \begin{equation}
        |f - f^{D_n}|_{1-var; [0,1]} = |f - I|_{1-var; [0,1]},
    \end{equation}
    where $I(x)=x.$ This gives an explicit proof that $f^D \not \rightarrow f$
\end{example}
\begin{proof}[Outline]
    Show the self-similarity of $f,$ and see the role of the piecewise approximation in an interval of $D_n$ as the role of the identity function in $[0,1].$
\end{proof}

\subsubsection{Closure of smooth paths in variation norm}

\begin{prop}\label{theo:1varL1}
    The map $y \mapsto \int_0^{\cdot} y_t dt$ is a Banach space isomorphism from 
    \begin{equation}
        L^1([0,T], \R^d) \rightarrow C_0^{0,1-var}([0,T], \R^d).
    \end{equation}
    The Banach space isometry $|x|_{1-var} = |\dot{x}|_L^1$ holds.
\end{prop}
\begin{proof}[Outline of the proof]
    Show the isometric property of the map as defined from $C^{\infty}$ to $C^{\infty}$ and extend to the closures in the two norms. 
\end{proof}
\begin{prop}[Radon-Nikodym]
Let $x: [0,T] \rightarrow \R^d$ be absolutely continuous. Then it can be written as $x_t = x_0 + \int_0^t \dot{x}_s ds$ for some $\dot{x} \in L^1([0,T], \R^d).$ As a consequence:
\begin{equation}
    C^{0,1-var}([0,T], \R^d) = \{ \text{absolutely continuous functions } [0,T] \rightarrow \R^d \}
\end{equation}
\end{prop}

\begin{corollary}
    Let $x \in C^{1-var}([0,T], \R^d).$ Then piecewise linear approximation converge in $1$-var norm:
    \begin{equation}
        |x - x^D|_{1-var; [0,T]} \rightarrow 0 \text{ as } |D| \rightarrow 0
    \end{equation}
    if and only if $x \in C^{0, 1-var}([0,T], \R^d).$ That is if and only if $x$ is absolutely continuous.
\end{corollary}
\begin{proof}[Outline of the proof]
    $\Leftarrow:$ Approximate  with a smooth path, prove the property for smooth paths and finally use triangular inequality.
\end{proof}

\begin{corollary}
 The space $C^{0,1-var}([0,T],\R^d)$ is a separable Banach space (and hence Polish).    
\end{corollary}
\begin{proof}[Outline of the proof]
    Take piecewise linear paths with rational values on the diadic partition $D^n = \{ k 2^{-n}T: k \leq 2^n \}.$
\end{proof}

\subsubsection{Lipschitz continuity}

\begin{prop}
    $C^{1-Hol}([0,T], \R^d)$ is Banach with norm $|x_0| + |x|_{1-Hol,[0,T]}.$ It is not separable.
\end{prop}

\begin{prop}
    The map $y \mapsto \int_0^{\cdot} y_t dt$ is a Banach space isomorphism from 
    \begin{equation}
        L^{\infty}([0,T], \R^d) \rightarrow C_0^{1-Hol}([0,T], \R^d).
    \end{equation}
    The Banach space isometry $|x|_{1-Hol} = |\dot{x}|_{L^{\infty}}$ holds.
\end{prop}

\begin{prop}[reparametrization]
    Let $x \in C^{1-var}([0,T], \R^d),$ not constant. Then let $y\phi = x,$ where 
    \begin{equation}
        \phi(t) = \frac{|x|_{1-var, [0,t]}}{|x|_{1-var, [0,T]}}.
    \end{equation}
    Then $y \in C^{1-Hol}([0,T], \R^d)$ has constant speed.
    \begin{equation}
        |\dot{y}(t)| \equiv |x|_{1-var, [0,T]} = |y|_{1-Hol,[0,1]}, \ a.e.\ t \in [0,T]
    \end{equation}
\end{prop}

\begin{prop}
    The closure of smooth paths in $C^{1-Hol}([0,T],\R^d)$ equals $C^1([0,T], \R^d).$
\end{prop}
\begin{proof}[Outline of the proof]
    Show that $C^1$ with $x \mapsto |x_0| + |\dot{x}|_{\infty}$ is Banach and that norm is equal to the 1-Holder norm. 
\end{proof}

\subsection{Sobolev spaces of continuous paths of bounded
variation}

\subsubsection{Paths of Sobolev regularity on $\R^d$}

\begin{prop}
    The space $W^{1,p}([0,T],\R^d)$ is a Banach space under the norm
    \begin{equation}
        x \mapsto |x_0| + |x|_{W^{1,p};[0,T]}
    \end{equation}
    They are separable if and only if $p \in [1, \infty).$
\end{prop}

\begin{lemma}
    An equivalent norm is 
    \begin{equation}
        |x|_{L^q} + |\dot{x}|_{L^p},
    \end{equation}
    for all $q \in [1,\infty].$
\end{lemma}

\begin{theorem}
    Let $p \in (1,\infty).$ Given $x \in W^{1,p}([0,T],\R^d),$
    \begin{equation}
        \omega(s,t) = |x|_{W^{1,p};[s,t]}(t - s)^{1 - 1/p}
    \end{equation}
    defines a control and $|x|_{1-var,[s,t]} \leq \omega(s,t).$ Hence we have a continuous embedding $W^{1,p} \hookrightarrow C^{1-var}.$
\end{theorem}
\begin{proof}[Outline of the proof]
    Use the Holder inequality on $\int_s^t |x_t| \cdot 1 dt$ to show that $\omega(s,t) \geq d(x_s, x_t),$ use it again in the sequence formulation to show superadditivity.
\end{proof}

\begin{prop}\label{theo:sobolevnorm}
    Let $p \in (1,\infty).$ A function $x: [0,T] \rightarrow \R^d$ is in $W^{1,p}$ if and only if $M_p(x) < \infty,$ where
    \begin{equation}
        M_p(x) := \sup_D \sum_i \frac{|x_{t_i t_{i+1}}|^p}{(t_{i+1} - t_i)^{p-1}}.
    \end{equation}
    In this case $M_p(x) = |x|^p_{W^{1,p};[0,T]}$
\end{prop}
\begin{proof}[Outline of the proof]
    $\Leftarrow$: if $M_p(x) < \infty,$ then prove that $x$ is absolutely continuous. 
    It follows that the piecewise linear approximations of $x$ converge to $x$ in 1-variation norm.
    Hence one can prove the property for PL approximations and conclude with Fatou's lemma.
\end{proof}

\subsubsection{Paths of Sobolev regularity on metric spaces}

For paths in a generic metric space $E$ Sobolev norms are defined using $M^p(\cdot)$ as in Proposition \ref{theo:sobolevnorm}. 

\begin{theorem}
    For any $x \in W^{1,p}([0,T], E)$ we have for all $s < t$
    \begin{equation}
        d(x_s, x_t) \leq |x|_{1-var, [s,t]} \leq |x|_{W^{1,p};[s,t]} (t - s)^{1 - 1/p}.
    \end{equation}
    In particular $W^{1,p} \subset C^{1-var}.$
\end{theorem}

\begin{prop}
    For every $x \in C([0,T],\R^d),$ 
\begin{equation}
    |x|_{W^{1,p};[0,T]}^p =\lim_{\delta \rightarrow 0} \sup_{(t_i) \in D_\delta ([0,T])} \sum_{i:t_i \in D} \frac{|d(x_{t_i}, x_{t_{i+1}})|^p}{|t_{i+1} - t_i|^{p-1}} \in [0, \infty].
\end{equation}
\end{prop}
\begin{proof}[Outline of the proof]
    It suffices to prove the super-additivity of the operator under summation, which follows from the convexity of the map $x \mapsto x^p.$
\end{proof}

\begin{example}
    Let $C([0,T], E)$ be equipped with the uniform topology. Let $p \in (1,\infty).$ Then:
    \begin{enumerate}
        \item show
        \begin{equation}
            x \in C([0,T], E) \mapsto M_p(x):= |x|_{W^{1,p};[0,T]} \in [0, \infty) 
        \end{equation}
        is lower semicontinuous.
        \item Assume $E$ has the Heine-Borel property (bounded sets have compact closure). Then level sets:
        \begin{equation}
            \left\{ x \in C_0([0,T], E): M_p(x) \leq \Lambda \right\} \text{ with } \Lambda \in [0, \infty) \text{ and } o \in E
        \end{equation}
        are compact (\textit{hint: use Ascoli-Arzelà}).
    \end{enumerate}
\end{example}

\section{Riemann-Stieltjes Integration}

\begin{definition}
    Let $x:[0,T] \rightarrow \R^d$ and $y:[0,T] \rightarrow L(\R^d, \R^e).$ Let $D^n$ be a sequence of dissections with $|D_n| \rightarrow 0$ and $\xi^n_i$ points in $[t_i, t_{i+1}].$ Assume that $\sum y(\xi^n_i) x_{t^n_i, t^n_{i+1}}$ converges when $n \rightarrow \infty$ to a limit $I$ independent of the $\xi^n_i$ and of $D_n.$ Then we say the $I$ is the Riemann-Stieltjes of $y$ against $x$ and indicate:
    \begin{equation}
        I =: \int_0^T y dx.
    \end{equation}
\end{definition}

\begin{prop}
    Let $x \in C^{1-var}([0,T],\R^d)$ and $y:[0,T] \rightarrow L(\R^d, \R^e)$ piecewise continuous. Then the R-S integral exists, is linear in $x$ and $y$ and:
    \begin{equation}\label{eq:RSEstimate}
        \left| \int_0^t y dx \right| \leq |y|_{\infty;[0,T]} |x|_{1-var; [0,T]}.
    \end{equation}
    Moreover
    \begin{equation}\label{eq:integralAdditivity}
        \int_s^t y_u dx_u = \int_0^t y_u dx_u - \int_0^s y_u dx_u.
    \end{equation}
\end{prop}
\begin{proof}[Outline of the proof]
    \begin{enumerate}
        \item Step functions,
        \item $dx-$integrable functions form a linear space,
        \item Estimate\ref{eq:RSEstimate},
        \item $dx-$integrable functions are closed in sup topology,
        \item $d(\alpha x + \beta \tilde{x})$-integrability and multilinearity,
        \item Continuous functions, approximating with step functions, piecewise-continuous functions,
        \item Equation \ref{eq:integralAdditivity}.
    \end{enumerate}
\end{proof}

\begin{exercise}
    Show that for $x \in C^1([0,T],\R^d),$ we have:
    \begin{equation}
        \int_0^T y_u dx_u = \int_0^T y_u \dot{x}_u du.
    \end{equation}
\end{exercise}

\begin{prop}[Integration by parts, seems wrong in the book]
    Let $x \in C^{1-var}([0,T],\R^d)$ and $y\in C^{1-var}([0,T], L(\R^d, \R^e)).$ Then:
    \begin{equation}
        \int_0^T y_u dx_u + \int_0^T x_u dy_u = x_T y_T - x_0 y_0.
    \end{equation}
\end{prop}
\begin{proof}[Interpretation]
    When $e=1,$ the space $L(\R^d, \R)$ is isomorphic to $\R^d$ via duality as metric spaces. 
    Hence the definition of the integral is clear.
    When $e>1,$ one can see $y = (y_1, \ldots, y_e)$ a collection of maps in $L(\R^d, \R)$ and define the integral as $\int x dy = (\int x dy_1, \ldots, \int x dy_e).$
    Note that $x(t)$ here is interpreted as an element of $L(\R^d, \R).$
    The proof is trivial using appropriate approximating sums.
\end{proof}

\begin{exercise}
    Take $x \in C^{1-var}([0,T_1, \R^d])$ and $y \in C^{1-var}([0,T_1, L(\R^d,\R^e)])$ and assume that $\phi$ is a non-decreasing continuous function from $[0,T_1]$ onto $[0,T_2].$ 
    \begin{equation}
        \int_0^t y\phi d(x\phi) = \int_0^{\phi(t)} y dx, \ \forall t \in [0,T_1].
    \end{equation} 
\end{exercise}

\begin{exercise}
    Let $x \in C^{1-var}([0,T], \R^d),$ let $\phi$ a $C^{\infty}$ function compactely supported on $[-1,1],$ and define $\Phi_t = \int_{-\infty}^t \phi(u)du$ and extend $x$ to $[-\infty, \infty]$ as $x_0$ below and $x_T$ above.
    The \textit{mollifier approximation} is:
    \begin{equation}
        x^\varepsilon: t \in [0,T] \mapsto x_0 + \int_{\R} \phi_{(t-s) / \varepsilon} dx_s.
    \end{equation}

    \begin{itemize}
        \item $x^{\varepsilon}$ is infinitely differentiable [\textit{derivative under the integral sign, follows from the fact that convergence to the derivative is uniform for $C^1$ functions}].
        \item $|x^\varepsilon|_{1-var,[0,T]} \leq |x|_{1-var,[0,T]}$ and $|x^\varepsilon|_{1-Hol,[0,T]} \leq |x|_{1-Hol,[0,T]}.$ [\textit{Use Fubini version between the Lebesgue measure $dt$, and the Radon measure $d|x_t|$}]
        \item $x^{\varepsilon}$ converges to $x$ in supremum topology as $\varepsilon \rightarrow \infty.$
    \end{itemize}

    Observe that using integration by parts we have:
    \begin{equation}
        x_t^\varepsilon = \int_{-1}^1 x_{t + \varepsilon s} \phi(s) ds.
    \end{equation}
\end{exercise}

\subsection{Continuity properties}

\begin{prop}[Continuity]
    Let $y^n, y: [0,T] \rightarrow L(\R^d, \R^e)$ be continuous functions and assume $y_n \rightarrow y$ uniformly. Assume $x^n, x \in C^{1-var}([0,T], \R^d)$ and $x^n \rightarrow x$ uniformly with
    \begin{equation}
        \sup_n |x^n|_{1-var, [0,T]} < \infty.
    \end{equation} 
    Then 
    \begin{equation}
        \int_0^t y^n dx^n \rightarrow \int_0^t y dx \text{ uniformly for } t \in [0,T].
    \end{equation}
\end{prop}

\begin{prop}[Uniform continuity on bounded sets]
   Let $y, y' \in C([0,T], L(\R^d, \R^e))$ and $x,x' \in C^{1-var, \R^d}.$ Then
\begin{equation}
    |\int_0^\cdot ydx - \int_0^\cdot y'dx'| \leq |x|_{1-var} |y - y'|_\infty + |x - x'|_{1-var} |y'|_\infty 
\end{equation} 
\end{prop}
\begin{proof}[Outline of the proof]
    Add and substract $\int_0^\cdot y'dx.$
\end{proof}

\begin{corollary}
     Let $y, y' \in C([0,T], \R^e)$ and $x,x' \in C^{1-var, \R^d}$ and $V: \R^e \rightarrow L(\R^d, \R^e)$ continuous. 
     Assume
     \begin{equation}
        |x|_{1-var}, |x'|_{1-var}, |y|_\infty, |y'|_\infty < R.
     \end{equation} 

     Then exist $\delta = \delta(\varepsilon, R, V)$ such that
     \begin{equation}
        |\int_0^\cdot V(y)dx - \int_0^\cdot V(y')dx'| \leq \varepsilon.
     \end{equation}
\end{corollary}

\section{Ordinary differential equations}

We consider equations of the type:
\begin{equation}
    dy = V(y) dx,
\end{equation}
where $y$ is a path in $\R^e,$ $x$ is a path in $\R^d$ and $V:\R^e \rightarrow L(\R^e, \R^d).$
Observe that $V$ can be seen as $V = (V_1, \ldots, V_d),$ where $V_i: \R^e \rightarrow \R^e.$
One can see the $V_i(y)$s as the columns of $V(y)$ seen as a matrix.
In this case:
\begin{equation}
    \int V(y) dx = \sum_i^d \int V_i(y) dx_i.
\end{equation} 

\begin{definition}
    A collection of vector fields $V = (V_1, \ldots, V_n) : \R^e \rightarrow L(\R^e, \R^d)$ is said \textit{bounded} if
    \begin{equation}
        |V|_\infty = sup_{y \in \R^e} |V(y)| < \infty.
    \end{equation}

    For any $U \subset \R^e,$ we define the $1$-Lipschitz norm as:
    \begin{equation}
        |V|_{Lip^1(U)} := \max \left\{ sup_{y \neq z,\ y,z \in U} \frac{|V(y) - V(z)|}{|y - z|}, sup_{y \in U} |V(y)| \right\}.
    \end{equation}

    We say that $V \in Lip^1(\R^e)$ if $|V|_{Lip^1(\R^n)} < infty$ and locally 1-Lipschitz if it is 1-Lipschitz in any bounded subset $U$ of $\R^e.$
\end{definition}

\begin{lemma}[Gronwall's lemma]
    Let $x \in C^{1-var}([0,T], \R^d)$ and $\phi:[0,T] \rightarrow \R^+$ a bounded measurable function. Assume that for all $t \in [0,T]$
    \begin{equation}\label{eq:gronwall}
        \phi_t \leq K + L \int_0^t \phi_s |dx_s|.
    \end{equation}

    Then for all $t \in [0,T]$
    \begin{equation}
        \phi_t \leq K exp\left( L \int_0^t |dx_s| \right).
    \end{equation}

    The same works substituting $K$ with a non-decreasing function $K_t.$
\end{lemma}
\begin{proof}[Outline of the proof]
    Apply inductively Equation \eqref{eq:gronwall} and use the exponential series definition.
    It also requires the iterated integral of $|dx_t|,$ which is easily solved with integration by parts.
\end{proof}

\subsection{Existence}

\begin{theorem}[Existence]
    Assume that
    \begin{itemize}
        \item $V = (V_1, \ldots, V_d)$ is a collection of continuous, bounded vector fields on $\R^e$,
        \item $y_0 \in \R^e$ is the initial condition,
        \item $x \in C^{1-var}([0,T], \R^d)$.
    \end{itemize}
    Then the exist a solution (not necessarily unique) of the ODE $dy = V(y)dx$.
    Moreover for all $s < t$,
    \begin{equation}
        |\pi_{(V)}(0, y_0; x)|_{1-var, [s,t]} \leq |V|_\infty |x|_{1-var, [s,t]}.
    \end{equation}
\end{theorem}
\begin{proof}[Outline of the proof]
    Define the Euler approximation on a dissection $D$.
    Bound the variations $y^D_{s,t}$ and consequently the $|y^D|_\infty.$
    Then by Ascoli Arzelà $y^{D_n}$ must have a limit point.
    Hence one can pass to the limit in the integral that defined $y^D_{s,t}$ by dominated convergence.
    Note that one has to pass to the limit in $V(y_{r_D}^D)$, where $r_D$ is the largest element in $D$ smaller than $r.$
\end{proof}

\begin{theorem}[Existence up to explosion time]
      Assume that
    \begin{itemize}
        \item $V = (V_1, \ldots, V_d)$ is a collection of continuous vector fields on $\R^e$,
        \item $y_0 \in \R^e$ is the initial condition,
        \item $x \in C^{1-var}([0,T], \R^d)$.
    \end{itemize}
    Then either there exists a solution (not necessarily unique) of the ODE $dy = V(y)dx$, $y:[0,T] \rightarrow \R^e,$
    or there exist a $\tau \in [0,T]$ and a local solution $y:[0,\tau) \rightarrow \R^e$ that solves in any $[0,t], \ t < \tau,$ and 
 \begin{equation}
    lim_{t \nearrow \tau} |y_t| = +\infty.
 \end{equation}
\end{theorem}
\begin{proof}[Outline of the proof]
    Consider the solution in balls around $y_0$ of increasing radius. Up to reaching the border at $\tau_i$ of each ball we have a solution by the Existence theorem.
    Proceeding inductively we either reach $T$ with the $\tau_i$ or we converge to a $\tau.$ In the second case there is explosion.
\end{proof}

\begin{theorem}[No explosion]
 Assume that
    \begin{itemize}
        \item $V = (V_1, \ldots, V_d)$ is a collection of continuous vector fields on $\R^e$ of \textbf{linear growth}, that is
        \begin{equation}\label{eq:lineargrowth}
             \exists A > 0:\ |V(y)| < A (1 + |y|),\ \forall y \in \R^e,
        \end{equation}
        \item $y_0 \in \R^e$ is the initial condition,
        \item $x \in C^{1-var}([0,T], \R^d)$ and $l \geq A |x|_{1-var,[0,T]}.$
    \end{itemize}   
    Then explosion cannot happen. Moreover any solution satisfies:
    \begin{equation}
        |y|_{\infty, [0,T]} \leq (|y_0| + l) \exp(l).
    \end{equation}

    Furthermore for all $0 \leq s \leq t \leq T,$ 
    \begin{equation}
        |y|_{1-var, [s,t]} \leq (1 + |y_0|) \exp(2l) A \int_s^t |dx_u|.
    \end{equation}
\end{theorem}
\begin{proof}[Outline of the proof]
    Use the condition in Equation \eqref{eq:lineargrowth} plus Gronwall to control $|y_s|.$
    Then use Gronwall again to control $|y_{u,v}|.$
\end{proof}

\subsection{Uniqueness}

\begin{theorem}
    Assume that 
    \begin{itemize}
        \item $V = (V_1, \ldots, V_d)$ is a collection of Lipschitz continuous vector fields on $\R^e,$ such that
        \begin{equation}
            v \geq \sup_{y,z \in \R^e} \frac{|V(y) - V(z)|}{|y - z|}.
        \end{equation}
        \item $x \in C^{1-var}([0,T],\R^d)$ such that $v |x|_{1-var,[0,T]} \geq l.$
    \end{itemize}
    Then, for every initial condition $y_0$ there exists a unique ODE solution and the associated flow is Lipschitz continuous, that is:
    \begin{equation}\label{eq:solutionLipschitz}
         |\pi_{(V)}(0, y^1_0; x) - \pi_{(V)}(0, y^2_0; x)|_\infty \leq |y^1_0 - y^2_0| \exp(l).
    \end{equation}
    Moreover, for all $s < t$ in $[0,T],$
    \begin{equation}
        |\pi_{(V)}(0, y^1_0; x) - \pi_{(V)}(0, y^2_0; x)|_{1-var,[s,t]} \leq |y^1_0 - y^2_0| \exp(2l) v |x|_{1-var,[s,t]}.
    \end{equation}
\end{theorem}
\begin{proof}[Outline of the proof]
    First use Gronwall on $\bar{y} = y^1 - y^2.$ This is enough to prove Equation \eqref{eq:solutionLipschitz}.
    Then use Gronwall again on $\bar{y}_{s,t}.$
\end{proof}

\subsection{Consequences of uniqueness}
\begin{prop}[Change of variable]
    Let $x \in C^{1-var}([0,T_1], \R^d)$ and $V = (V_1, \ldots, V_d)$ a collection of locally Lipschitz vector fields on $\R^e$ of linear growth.
    Assume $\phi$ is a non-decreasing function from $[0,T_2]$ onto $[0,T_1].$ so that
\begin{equation}
    x \circ \phi \in C^{1-var}([0,T_2], \R^d).
\end{equation}
Then 
\begin{equation}
    \pi_{(V)}(0, y_0; x)_{\phi(\cdot)} = \pi_{(V)}(0, y_0; x \circ \phi) \text{ on } [0,T_2].
\end{equation}
\end{prop}

\subsection{Continuity of the solution map}

\subsubsection{Limit theorem for 1-variation signals}

\begin{theorem}
    We consider
    \begin{itemize}
        \item $V^1 = (V_1^1, \ldots V^1_d)$ and $V^2 = (V_2^1, \ldots V^2_d)$ two collections of $Lip^1$ vector fields on $\R^e$ such that
        \begin{equation}
            \max_{i=1,2} |V^i|_{Lip^1} \leq v.
        \end{equation}
        \item $y_0^1$ and $y_0^2$ initial conditions.
        \item $x^0$ and $x^1$ are paths in $C^{1-var}([0,T], \R^d)$ such that 
        \begin{equation}
            \max_{i=1,2} |x^i|_{1-var, [0,T]} \leq l.
        \end{equation} 
    \end{itemize}
    Then, if $y^i = \pi_{V_i}(0, y_0^i; x^i)$ we have\footnote{Recall that $|x|_0 = \sup_{u,v \in [0,T]} |x_u - x_v|$}.
    \begin{equation}
        |y^1 - y^2|_\infty \leq \left( |y_0^1 - y_0^2| + v |x^1 - x^2|_0 + |V^1 - V^2|_\infty l \right) \exp(2vl).
    \end{equation}
\end{theorem}
\begin{proof}[Outline of the proof]
    Add and subtract, integration by parts and finally Gronwall.
\end{proof}

\begin{corollary}[Localization]
      We consider
    \begin{itemize}
        \item $V^1 = (V_1^1, \ldots V^1_d)$ and $V^2 = (V_2^1, \ldots V^2_d)$ two collections of locally $Lip^1$ vector fields on $\R^e$ with linear growth,
        \item $y_0^1$ and $y_0^2$ initial conditions, with $R \geq |y_0^i|,$
        \item $x^0$ and $x^1$ are paths in $C^{1-var}([0,T], \R^d)$ such that 
        \begin{equation}
            \max_{i=1,2} |x^i|_{1-var, [0,T]} \leq l.
        \end{equation} 
    \end{itemize}
    Then, if $y^i = \pi_{V_i}(0, y_0^i; x^i)$
     \begin{equation}
        |y^1 - y^2|_\infty \leq C \left( |y_0^1 - y_0^2| + |x^1 - x^2|_0 + |V^1 - V^2|_{\infty, B(0,M)} \right).
    \end{equation}
\end{corollary}

\begin{exercise}[Change of variable formula]
    Change of variable.
\end{exercise}

\subsubsection{Continuity under the 1-variation norm}

\begin{theorem}
    We consider
    \begin{itemize}
        \item $V^1 = (V_1^1, \ldots V^1_d)$ and $V^2 = (V_2^1, \ldots V^2_d)$ two collections of $Lip^1$ vector fields on $\R^e$ such that
        \begin{equation}
            \max_{i=1,2} |V^i|_{Lip^1} \leq v,
        \end{equation}
        \item $y_0^1$ and $y_0^2$ initial conditions, with $R \geq |y_0^i|,$
        \item $x^1$ and $x^2$ are paths in $C^{1-var}([0,T], \R^d)$ such that 
        \begin{equation}
            \max_{i=1,2} |x^i|_{1-var, [0,T]} \leq l.
        \end{equation} 
    \end{itemize}

    Then, if $y^i = \pi_{V_i}(0, y_0^i; x^i)$
    \begin{equation}
        |y^1 - y^2|_{1-var,[0,T]} \leq 2 \left( |y_0^1 - y_0^2| v l + |x^1 - x^2|_{1-var,[0,T]} + |V^1 - V^2|_\infty l \right) e^{3vl}.
    \end{equation}
\end{theorem}

\begin{corollary}[Localization]
        We consider
    \begin{itemize}
        \item $V^1 = (V_1^1, \ldots V^1_d)$ and $V^2 = (V_2^1, \ldots V^2_d)$ two collections of locally $Lip^1$ vector fields on $\R^e,$ with linear growth,
        \item $y_0^1$ and $y_0^2$ initial conditions, with $R \geq |y_0^i|,$
        \item $x^1$ and $x^2$ are paths in $C^{1-var}([0,T], \R^d)$ such that 
        \begin{equation}
            \max_{i=1,2} |x^i|_{1-var, [0,T]} \leq l.
        \end{equation} 
    \end{itemize}

    Then, if $y^i = \pi_{V_i}(0, y_0^i; x^i)$ there exist constants $C, M$ depending only on $R, l$ and the vector fields $V$ such that
    \begin{equation}
        |y^1 - y^2|_{1-var,[0,T]} \leq C \left( |y_0^1 - y_0^2| + |x^1 - x^2|_{0,[0,T]} + |V^1 - V^2|_{\infty, B(0,M)} \right).
    \end{equation}
\end{corollary}

\subsection{Summary of the regularity estimates for ODEs}
Summarize in a scheme conditions on $V$ and their consequence on $y.$

\begin{itemize}
    \item \textit{existence}: $V$ continuous, bounded $\Rightarrow$ Global existence and $1-var$ estimate.
    \item \textit{explosion}: $V$ continuous $\Rightarrow$ Existence up to explosion time $\tau$ or global.
    \item \textit{non explosion}: $V$ continuous, linear growth $\Rightarrow$ No explosion, uniform and $1-var$ estimates.
    \item \textit{uniqueness}: $V$ globally $Lip^1$ $\Rightarrow$ esistence and uniqueness, continuity wrt $y^0$ in uniform norm and $1-var.$
    \item \textit{local uniqueness}: $V$ locally $Lip^1$ $\Rightarrow$ esistence and uniqueness up to explosion time. If no explosion (e.g. linear growth), global unique solution.
    \item \textit{uniform continuity}: $V^i$ globally $Lip^1$ $\Rightarrow$ uniform continuity wrt $y_0,\ V,\ x$.
    \item \textit{uniform local continuity}: $V^i$ locally $Lip^1$, linear growth $\Rightarrow$ continuity wrt $y_0,\ V,\ x$, with estimates depending on $|x|_{1-var},\ |y_0|,\ V.$
    \item \textit{1-var continuity}: $V^i$ globally $Lip^1$ $\Rightarrow$ $1-var$ continuity wrt $y_0,\ V,\ x$.
    \item \textit{local 1-var continuity}: $V^i$ locally $Lip^1$, linear growth $\Rightarrow$ $1-var$ continuity wrt $y_0,\ V,\ x$, with estimates depending on $|x|_{1-var},\ |y_0|,\ V.$
\end{itemize}

That is, boundedness or linear growth implies global existence, continuity implies existence up to explosion, global $Lip^1$ implies global uniqueness and continuity (uniform and in $1-var$), local $Lip^1$ implies uniqueness up to explosion and continuity (uniform and in $1-var$).
\section{ODE: Smoothness}

\begin{definition}
    We say that $V: \R^e \rightarrow L(\R^d, \R^e)$ is $C^k$-bounded if
    \begin{itemize}
        \item it is $k$ times Frechet differentiable,
        \item $(V, DV, \ldots, D^k V)$ is a bounded function on $\R^e.$
    \end{itemize}
    We set
    \begin{equation}
        |V|_{C^k} := \max_{i=0, \ldots k} |D^k V|_\infty.
    \end{equation}
    In case only the first condition holds we say $V \in C^k_{loc}.$
\end{definition}

\subsubsection{Directional derivatives}

\begin{lemma}
    Let $V = (V_1, \ldots, V_d)$ continuously differentiable vector fields on $\R^e,$ $V \in C^1(\R^e, L(\R^d,\R^e)).$
    Then for all $\varepsilon > 0,$ for all $U \subseteq \R^e$ bounded, there exists $\delta$ such that if $a,b \in U$ and $|a-b|<\delta$, 
    \begin{equation}
        |V(b) - V(a) - DV(a) \cdot (b - a)| \leq \varepsilon |b - a|. 
    \end{equation}
\end{lemma}

\begin{theorem}[Directional derivatives]\label{theo:directionalRS}
    Fix a collection of $C^1_{loc}$ vector fields $V$ satisfies the non-explosion condition.
    Then
    \begin{itemize}
        \item the map
        \begin{equation}
            (y_0, x) \mapsto \pi(0, y_0, x) 
        \end{equation}
        has directional derivatives
        \begin{equation}
            D_{(v,h)}\pi(0,y_0,x) := \left\{ \frac{d}{d\varepsilon} \pi(0, y_0 + \varepsilon v, x + \varepsilon h) \right\}_{\varepsilon = 0} \in C^{1-var}([0,T], \R^e).
        \end{equation}
        in all directions $(v,h) \in \R^e \times C^{1-var}$.
        \item define the bounded variation paths
        \begin{equation}
            M_t = \sum \int_0^t DV_i(y_s) dx^i_s
        \end{equation}
        and 
        \begin{equation}
            H_t = \sum \int_0^t V_i(y_s) dh^i_s.
        \end{equation}
        Then $z = D_{(v,h)}\pi(0,y_0,x)$ is the unique solution of:
        \begin{equation}
            \begin{cases}
                dz_t = dM_t \cdot z_t + dH_t,\\
                z_0 = v.
            \end{cases}
        \end{equation}
    \end{itemize}
\end{theorem}
\begin{proof}[Outline of the proof]
    By a localisation argument we can assume $V$ is compactly supported.
    Then take $y^\varepsilon = \pi(0, y_0 + \varepsilon v, x + \varepsilon h)$ and $z^\varepsilon = \frac{y^\varepsilon - y}{\varepsilon}.$
    We want to prove that $z^\varepsilon$ converges to $z$ uniformly.
    We can bound $|z^\varepsilon|_{1-var}$ using the continuity in $1-var$ explicit estimate ($C^1$ implied $Lip^1.$)
    We can bound also $|y_t^\varepsilon$ uniformly in $t$ and $\varepsilon \in [0,1].$
    Then split $z^\varepsilon - z$ integral form and estimate each piece using $V \in Lip^1$ and the Lemma.
    Finally use Gronwall.
    To prove that the limit is actually in $1-var$ we use the following Proposition.
\end{proof}

\begin{prop}[Solution of the ODE in Theorem \ref{theo:directionalRS}]
    Consider the ODE:
    \begin{equation}\label{eq:directionalODE}
        dz = dM \cdot z + dH, 
    \end{equation}
    where we defined $dM = DV(y) dx$ and $dH = V(y) dh.$
    Then Equation\ref{eq:directionalODE} admits a unique solution $z$ under the hypothesis of Theorem \ref{theo:directionalRS}.
\end{prop}
\begin{proof}[Outline of the proof]
    Consider the homogeneous solution:
    \begin{equation}
        d \Phi = dM \cdot d \Phi.
    \end{equation}
    with initial condition $\Phi = I.$
    Then consider $z_t = \Phi_t \int_0^t \Phi_s^{-1} dH_s.$
    Observed that $\Phi$ is invertible as it is a matrix exponential.
\end{proof}

\begin{prop}[Higher order directional derivatives]
    Let $k \in \{ 1, 2, \ldots \}.$
    Assume $V$ is a collection of $C^k_{loc}$ vector fields on $\R^e$ satisfying the non-explosion condition. Then
    \begin{equation*}
        (y_0, x) \mapsto \pi_{(V)}(0, y_0, x)
    \end{equation*}
    has up to $k^{th}$ order directional derivatives in the sense that, for all $(v_i, h_i)_{1 \leq i \leq k} \in (\R^e \times C^{1-var}([0,T], \R^d))^{\times k},$
    \begin{equation}
        D_{(v_i, h_i)_{1 \leq k \leq k}}^k \pi(0, y_0, x) := \left\{ \frac{\partial^k}{\partial \varepsilon_1 \ldots \partial \varepsilon_k} \pi \left( 0, y_0 + \sum_{j=1}^k \varepsilon_j v_j, x + \sum_{j=1}^k \varepsilon_j h_j \right) \right\}_{\epsilon = 0}
    \end{equation}
    exists as a strong limit in the Banach space $C^{1-var}([0,T], \R^e).$
    Furthermore, the directional derivatives satisfy the control ODEs obtained from formal differentiation.
\end{prop}

\subsection{Frechet differentiability}

\begin{theorem}
    Let $V$ be a collection of $C^1_{loc}$ vector fields on $\R^e$ satisfying the non-explosion condition. Then the map
    \begin{equation}
        (y_0, x) \in \R^e \times C^{1-var}([0,T], \R^d) \mapsto y \equiv \pi(0, y_0, x) \in C^{1-var}([0,T], \R^e).
    \end{equation}
    is $C^1$ in the Frechet sense.
\end{theorem}
\begin{proof}[Outline of the proof]
    Use Proposition \ref{theo:FrechetDirectional} and the continuity results on $ODE$ solutions and the Riemann-Stieltjes integral.
\end{proof}
\appendix

\section{Banach spaces}

\begin{prop}[Derivative in embedded space]
    Assume $E \hookrightarrow F$ (continuous embedding). Assume $f \in C^1((a,b), F)$ is such that its derivative $\dot{f}$ exists as a strong limit in $F$ but actually takes values in $E$ and extends continuously to a function from $[a,b]$ into $E$. Then $f \in C^1((a,b), E)$ with derivative given by $\dot{f}.$
\end{prop}
\begin{proof}[Outline of the proof]
    Use fundamental theorem of calculus and observe the approximating Riemann sums are in $E$.
\end{proof}

Consider now $L(E,F)$ (linear continuous maps or linear bounded maps equivalently) is a Banach space with the operator norm.

\begin{definition}[Frechet derivative]
    Let $V, W$ normed vector spaces and $U \subseteq V$ open set. 
    A function $f:U \rightarrow W$ is called \textit{Frechet differentiable} at $x \in U$ if there exists a
    bounded linear operator $Df(x): V \rightarrow W$ such that
    \begin{equation}
        \lim_{|h|_V \rightarrow 0} \frac{|f(x + h) - f(x) - Df(x) h|_W}{|h|_V} = 0.
    \end{equation}

    It is said to be Frechet-differentiable on $U$ if $Df(x)$ exists for all $x \in U.$
    If $x \mapsto Df(x)$ is continuous, then we say $f$ is $C^1$ in the Frechet sense and we write $f \in C^1(U,F).$
\end{definition}

We want to see when the existence of directional derivatives is sufficient to determine Frechet differentiability.

\begin{example}[Counterexample with directional derivatives but not Frechet differentiable]
    \begin{equation}
        \begin{cases}
            f(0,0) = 0 \\
            f(x,y) = \frac{x^2 y}{x^2 + y^2}. 
        \end{cases}
    \end{equation}
\end{example}

\begin{prop}
    Let $U$ be an open set in $E$ and $f:U \rightarrow F$ a function that has directional derivatives in all directions. Let $A: U \rightarrow L(E,F)$ a continuous map such that
    \begin{equation}
        D_h f(x) = A(x) h.
    \end{equation}
    for all $x \in U,\ h \in E.$
    Then $f \in C^1(U,F)$ and $Df(x) h = A(x) h.$
\end{prop}
\begin{proof}[Outline of the proof]
    Use the fundamental theorem of calculus on $f(x + h) - f(x).$
\end{proof}

\begin{prop}\label{theo:FrechetDirectional}
    Let $U$ be an open set in $E$ and $f: U \rightarrow F$ be a continuous map that admits directional derivatives in all points and all directions. Let also
    \begin{equation}
        (x,h) \in U \times E \mapsto D_h f(x) \in F
    \end{equation}
    be uniformly continuous on bounded sets. Then $f$ is $C^1$ in the Frechet sense.
\end{prop}
\begin{proof}[Outline of the proof]
    Prove that $D_h f(x)$ is linear in $h$ using the fundamental theorem of calculus.
    Then use Lemma \ref{lem:Frechet}.
\end{proof}

\begin{lemma}\label{lem:Frechet}
    Let $U \subset E$ and $\phi: U \times E \rightarrow F$ uniformly continuous on bounded sets, such that for all $x \in U$ the map
    \begin{equation}
        h  \mapsto \phi(x,h) =: \phi(x) h.  
    \end{equation}
    is linear.

    Then the map
    \begin{align}
        &\tilde{\phi}: U \rightarrow L(E,F) \\
        &x \mapsto (h \mapsto \phi(x) h)
    \end{align}
    is well defined and uniformly continuous on bounded sets.
\end{lemma}
\end{document}

