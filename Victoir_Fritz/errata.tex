\documentclass[12pt]{article}
\usepackage[utf8]{inputenc}

\begin{document}

\section*{Errata for \emph{Multidimensional Stochastic Processes as Rough Paths} (Friz–Victoir)}

\begin{itemize}

\item \textbf{Proposition 1.21, proof, page \_\_}:  
In the proof, after defining the greedy subsequence
\[
j_0 = 0, \quad j_k = \max\{ i \ge j_{k-1} : t_i - t_{j_{k-1}} \le \delta \},
\]
it is claimed that the resulting subpartition has at most $\lfloor T/\delta \rfloor + 1$ intervals.
While the mesh bound $\le \delta$ is correct, the counting bound need not hold without additional assumptions.

\emph{Counterexample:}  
Take $\delta = 1$, $T = 3$, and
\[
D = \{ 0,\, 2t,\, 1+t,\, 1+3t,\, 2+2t,\, 3 \}, \quad 0 < t \ll 1.
\]
The greedy algorithm selects
\[
0 \to 2t \to 1+t \to 1+3t \to 2+2t \to 3,
\]
which yields $5$ intervals, exceeding $\lfloor 3/1 \rfloor + 1 = 4$.

This does not invalidate the proposition itself, but the intermediate bound in the proof should be revised.
A correct universal bound for the greedy algorithm is $2\lfloor T/\delta \rfloor + 1$.

\end{itemize}

\end{document}
