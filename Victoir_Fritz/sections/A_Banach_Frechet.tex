\section*{Banach spaces}

\begin{prop}[Derivative in embedded space]
    Assume $E \hookrightarrow F$ (continuous embedding). Assume $f \in C^1((a,b), F)$ is such that its derivative $\dot{f}$ exists as a strong limit in $F$ but actually takes values in $E$ and extends continuously to a function from $[a,b]$ into $E$. Then $f \in C^1((a,b), E)$ with derivative given by $\dot{f}.$
\end{prop}
\begin{proof}[Outline of the proof]
    Use fundamental theorem of calculus and observe the approximating Riemann sums are in $E$.
\end{proof}

Consider now $L(E,F)$ (linear continuous maps or linear bounded maps equivalently) is a Banach space with the operator norm.

\begin{definition}[Frechet derivative]
    Let $V, W$ normed vector spaces and $U \subseteq V$ open set. 
    A function $f:U \rightarrow W$ is called \textit{Frechet differentiable} at $x \in U$ if there exists a
    bounded linear operator $Df(x): V \rightarrow W$ such that
    \begin{equation}
        \lim_{|h|_V \rightarrow 0} \frac{|f(x + h) - f(x) - Df(x) h|_W}{|h|_V} = 0.
    \end{equation}

    It is said to be Frechet-differentiable on $U$ if $Df(x)$ exists for all $x \in U.$
    If $x \mapsto Df(x)$ is continuous, then we say $f$ is $C^1$ in the Frechet sense and we write $f \in C^1(U,F).$
\end{definition}

We want to see when the existence of directional derivatives is sufficient to determine Frechet differentiability.

\begin{example}[Counterexample with directional derivatives but not Frechet differentiable]
    \begin{equation}
        \begin{cases}
            f(0,0) = 0 \\
            f(x,y) = \frac{x^2 y}{x^2 + y^2}. 
        \end{cases}
    \end{equation}
\end{example}

The following Proposition proves that what is missing for a function with all the directional derivatives $D_h f(x)$ to be Frechet differentiable is linearity in $h$ and continuity in $x$.

\begin{prop}[Directional derivatives as a continuous linear map]\label{theo:FrechetDirectional1}
    Let $U$ be an open set in $E$ and $f:U \rightarrow F$ a function that has directional derivatives in all directions. Let $A: U \rightarrow L(E,F)$ a continuous map such that
    \begin{equation}
        D_h f(x) = A(x) h.
    \end{equation}
    for all $x \in U,\ h \in E.$
    Then $f \in C^1(U,F)$ and $Df(x) h = A(x) h.$
\end{prop}
\begin{proof}[Outline of the proof]
    Use the fundamental theorem of calculus on $f(x + h) - f(x).$
\end{proof}

A sufficient condition for Frechet differentiability is that $D_h f(x)$ is uniformly continuous on bounded sets as a function of $(h,x).$

\begin{prop}[Frechet differentiability condition]\label{theo:FrechetDirectional}
    Let $U$ be an open set in $E$ and $f: U \rightarrow F$ be a continuous map that admits directional derivatives in all points and all directions. Let also
    \begin{equation}
        (x,h) \in U \times E \mapsto D_h f(x) \in F
    \end{equation}
    be uniformly continuous on bounded sets. Then $f$ is $C^1$ in the Frechet sense.
\end{prop}
\begin{proof}[Outline of the proof]
    Prove that $D_h f(x)$ is linear in $h$ using the fundamental theorem of calculus.
    Then use Lemma \ref{lem:Frechet}.
\end{proof}

The following lemma is useful to connect the hypothesis of Proposition \ref{theo:FrechetDirectional} with that of Proposition \ref{theo:FrechetDirectional1}.

\begin{lemma}\label{lem:Frechet}
    Let $U \subset E$ and $\phi: U \times E \rightarrow F$ uniformly continuous on bounded sets, such that for all $x \in U$ the map
    \begin{equation}
        h \mapsto \phi(x,h) =: \phi(x) h.  
    \end{equation}
    is linear.

    Then the map
    \begin{align}
        &\tilde{\phi}: U \rightarrow L(E,F) \\
        &x \mapsto (h \mapsto \phi(x) h)
    \end{align}
    is well defined and uniformly continuous on bounded sets.
\end{lemma}
\end{document}

