\section{Variation and Holder spaces}

\begin{proposition}
    Assume $x: [0,T] \rightarrow E$ is $\alpha$-Holder continuous, with $\alpha 1$ or continuous with finite $p$-variation with $p \in (0, 1).$
    Then $x$ is constant, $x(\cdot) \equiv x_0.$
\end{proposition}

\begin{proposition}
    Let $x \in C([0,T], E)$. Then if $1 \leq p \leq p' < \infty,$
    \begin{equation}
        |x|_{p'-var, [0,T]} \leq |x|_{p-var, [0,T]}. 
    \end{equation} 
    Furthermore, if $0 \leq \alpha \leq \alpha' \leq 1,$
    \begin{equation}
        |x|_{\alpha-Hol,[0,T]} \leq |x|_{\alpha'-Hol,[0,T]} \cdot T^{\alpha' - \alpha}.
    \end{equation}
\end{proposition}

\begin{proposition}[Interpolation]
        Let $x \in C([0,T], E)$. Then if $1 \leq p \leq p' < \infty,$
    \begin{equation}
        |x|_{p'-var, [0,T]} \leq |x|_{p-var, [0,T]}^{p/p'} \cdot |x|_{0,[0,T]}^{1 - p/p'}. 
    \end{equation} 
    Furthermore, if $0 \leq \alpha' \leq \alpha \leq 1,$
    \begin{equation}
        |x|_{\alpha-Hol} \leq |x|_{\alpha'-Hol}^{\alpha' / \alpha} \cdot |x|_{0,[0,T]}^{1 - \alpha' / \alpha}.
    \end{equation}
\end{proposition}

\begin{proposition}
    Let $p \geq 1$ and $x \in C([0,T], E)$.
    \begin{itemize}
        \item $x \in C^{p-var}([0,T], E)$ is equivalent to
        \begin{equation}
            \lim_{\delta \rightarrow 0} \sup_{D_\delta} \sum_i d(x_{t_i}, x_{t_{i+1}})^p < \infty.
        \end{equation}
        \item If $1 \leq q < p < \infty$ and $x \in C^{q-var}([0,T], E),$ then
        \begin{equation}
            \lim_{\delta \rightarrow 0} \sup_{D_\delta} \sum_i d(x_{t_i}, x_{t_{i+1}})^p = 0.
        \end{equation}
    \end{itemize}
\end{proposition}
\begin{proof}[Outline of the proof]
    \begin{itemize}
        \item Bound the contribution of intervals with size larger then $\delta$ in $\sum d(x_{t_i}, x_{t_{i+1}})^p$ using their maximum and their quantity.
        \item Use the convergence to zero of the modulus of continuity 
        \begin{equation}
            osc(x, \delta) = \sup \{ d(x_s, x_t): s,t \in [0,T], |t - s| \leq \delta \}.
        \end{equation}
        as $\delta \rightarrow 0.$
    \end{itemize}
\end{proof}

\begin{proposition}[Control function from $p$-variation]
    Let $(E,d)$ be a metric space, $p \geq 1$ and $x: [0,T] \rightarrow E$ be a continuous path of bounded $p$-variation. Then
    \begin{equation}
        \omega_{x,p}(s,t) = |x|_{p-var}^p
    \end{equation}
    defines a control.
\end{proposition}
\begin{proof}
    This Proposition generalized Proposition \ref{theo:1varcontrol}.
    Superadditivity works similar to $p=1$.
    Continuity from inside is similar to $p=1$.
    We prove continuity from outside.
    \begin{itemize}
        \item $\omega_{x,p}(t,t^+) = 0$.
        We wish to negate the hypothesis $\omega_{x,p}(t,t^+) = \delta > 0$.
        It can be seen by choosing an $h_1$ such that if $h \leq h_1$, $d(x_t,x_{t+h}) < \delta / 8.$
        Then one can take a dissection of $[0,h_0]$ where $h_0 \leq h_1$.
        The sum of $d(x_{t_i},x_{t_{i+1}})^p$ can be bounded from below from the hypothesis.
        This yields a bound from below of the sum of $d(x_{t_i},x_{t_{i+1}})^p$ even excluding the first one.
        
        One can do the same on the first interval (dissect, bound the sum of $d(x_{t_i},x_{t_{i+1}})^p$ excluding the first one from below).

        This yields two non-overlapping intervals where the sum of $d(x_{t_i},x_{t_{i+1}})^p$ is more than $3/4 \delta.$ This yields $\omega(t, t^+) \geq 3/2 \delta.$ Since $h_0$ can be chosen small at will, this contradicts the hypothesis.
        \item Fix $s < t$ and a dissection of $[s,t+h]$ such that the $p$ power sum is higher than $\omega_{x,p}(s,t+h) - \varepsilon.$ Splitting the dissection as a dissection of $[s,t]$ and the rest and sending $h$ to zero yields the thesis.
    \end{itemize}
\end{proof}

\begin{proposition}
    Let $(E,d)$ be a metric space, $p \geq 1$ and $x$ a continuous path of finite $p$-variation and $\delta > 0.$ Then
    \begin{itemize}
        \item \begin{equation}
            \omega_{x,\delta,p}(s,t) \leq |x|_{p-var,[s,t]}^p
        \end{equation}
        defines a control.
        \item \begin{align}
            \omega_{x,\delta,p}(s,t) &= \sup_{D_\delta} \sum_i |x|_{p-var,[t_i,t_{i+1}]}^p \\
            \sup_{|t - s| \leq \delta} \frac{d(x_s, x_t)}{|t - s|^{1/p}} &= \sup_{|t - s| < \delta} |x|_{1/p-Hol, [s,t]}
        \end{align}
    \end{itemize}
\end{proposition}

\begin{proposition}\label{prop:pVarControl}
    Let $(E,d)$ be a metric space, $\omega$ a control, $p \geq 1$, $C>0$ and $x:[0,T] \rightarrow E$ a continuous path.
    \begin{itemize}
        \item The pointwise estimate 
        \begin{equation}
            d(x_s, x_t) \leq C \omega(s,t)^{1/p} \text{ for all } s < t \text{ in } [0,T].
        \end{equation}
        implies the $p$-variation estimate
        \begin{equation}
            |x|_{p-var, [s,t]} \leq C \omega(s,t)^{1/p} \text{ for all } s < t \text{ in } [0,T].
        \end{equation}
        \item Under the weaker assumption 
        \begin{equation}
            d(x_s, x_t) \leq C \omega(s, t)^{1/p} \text{ for all } s < t \text{ in } [0,T] \text{ such that } \omega(s,t) \leq 1.
        \end{equation}
        we have
        \begin{equation}
            |x|_{p-var, [s,t]} \leq 2 C \left( \omega(s,t)^{1/p} \vee \omega(s,t) \right) \text{ for all } s < t \text{ in } [0,T].
        \end{equation}
    \end{itemize}

\end{proposition}

\begin{corollary}[Not in the book, generalized version of Lemma\ref{lem:HolderControl}]\label{lem:pHolderControl}
    If $x \in C^{1/p-Hol}([0,T], E)$, then we have:
    \begin{equation}
        |x|_{p-var, [s,t]} \leq |x|_{1-Hol,[s,t]} |s - t|^{1/p}.
    \end{equation}
\end{corollary}
\begin{proof}
    Apply Proposition\ref{prop:pVarControl} with $\omega(u,v) = |u - v|$ and $C = |x|_{1-Hol,[s,t]}$ on the interval $[s,t]$.
    It implies that $|x|_{p-var, [u,v]} \leq |x|_{1-Hol,[s,t]} |u - v|^{1/p}|$ for all $u, v$ in $[s,t]$, hence the thesis.
\end{proof}

\begin{lemma}[Lower semicontinuity of $p$-var and $1/p$-Hol]
    Let $x^n$ be a sequence of paths from $[0,T]$ in $E$ of finite $p$-variation. Then
    \begin{equation}
        |x|_{p-var,[s,t]} \leq \liminf_{n \rightarrow \infty} |x^n|_{p-var, [s,t]}.
    \end{equation}
    In particular 
    \begin{equation}
        |x|_{1/p-Hold,[s,t]} \leq \liminf_{n \rightarrow \infty} |x^n|_{1/p-Hol, [s,t]}.
    \end{equation}
\end{lemma}

\begin{lemma}
    Let $x:[0,T] \rightarrow E$ be a continuous map of finite $p$-variation. Then, for all $s < t$, the map $p':[p,\infty) \mapsto |x|_{p'-var,[s,t]}$ is non-increasing and 
    \begin{equation}
        \lim_{p' \searrow p} |x|_{p'-var, [s,t]} = |x|_{p-var, [s,t]}.
    \end{equation}
\end{lemma}
\begin{proof}[Outline of the proof]
    We need to show that 
    \begin{equation}
    \omega(s,t) = \left( \lim_{p' \searrow p} |x|_{p'-var, [s,t]} \right)^p = \lim_{p' \searrow p} |x|_{p'-var, [s,t]}^{p'}
    \end{equation}
    is a control.
\end{proof}

\subsubsection{On some path spaces contained in $C^{p-var}([0,T], E)$}

\begin{observation}
    $x$ is a path of finite $p$-variation controlled by $(s,t) \mapsto |t - s|$ if an only if it is $1/p$-Holder.
\end{observation}

We show that conversely, any finite $p$-variation path can be reparametrized to a $1/p$-Holder path.

\begin{proposition}
    Let $(E,d)$ be a metric space and $x:[0,T] \rightarrow E$ a continuous path.
    Then $x$ is of finite $p$-variation if and only if there exists $h$ continuous increasing from $[0,T]$ to $[0,1]$ and $y:[0,1]\rightarrow E$ $1/p$-Holder, such that $x = y \circ h.$
\end{proposition}

\subsection{Approximation in Geodesic spaces}

\begin{definition}
    In a metric space $(E,d)$ a Geodesic joining two points $a,b \in E$ is a path $\Upsilon^{a,b}:[0,1] \rightarrow E$ such that $\Upsilon^{a,b}(0) = a,\ \Upsilon^{a,b}(1) = b$ and 
    \begin{equation}
        d \left( \Upsilon^{a,b}_s, \Upsilon^{a,b}_t \right) = |t - s| d(a,b).
    \end{equation} 
    If any two points are connected by a geodesic, we say that $E$ is a geodesic space.
\end{definition}

\begin{example}
    The unit circle $S^1$ in $\R^2$ with the induced metric is not geodesic. It is geodesic with the arclength distance.
\end{example}

Geodesic spaces have the structure that allows to generalize piecewise linear approximations.

\begin{definition}[PL approximations]
    $x$ continuous path in $E$, $D$ dissection.
    \begin{equation}
        \begin{cases}
            x^D_t = x_t \text{ for } t \in D \\
            x^D_t = \Upsilon^{x_{t_i}, x_{t_{i+1}}} \left( \frac{t - t_i}{t_{i+1} - t_i} \right) \text{ for } t \in (t_i, t_{i+1}). 
        \end{cases}
    \end{equation}
\end{definition}

\begin{lemma}
    $E$ geodesic space, $x \in C([0,T],E).$ Then $x^D$ converges uniformly to $x$ on $[0,T].$
\end{lemma}

\begin{proposition}
    Let $E$ be a geodesic space and $x \in C^{p-var}([0,T], E)$, $p \ge 1$ and $D$ a dissection of $[0,T]$. Then
    \begin{equation}
        |x^D|_{p-var,[0,T]} \le 3^{1-1/p} |x|_{p-var, [0,T]}.
    \end{equation}

    If $x$ is $1/p$-Holder,
    \begin{equation}
        |x^D|_{1/p-Holder,[0,T]} \le 3^{1-1/p} |x|_{1/p-Hol, [0,T]}.
    \end{equation}
\end{proposition}